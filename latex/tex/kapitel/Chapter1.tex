%! Author = Eyüp Ünlü
%! Date = 25/10/2021

\chapter{Exposé}\label{ch:exposé}


\section{Importance of the topic}
Robotic Process Automation (RPA) enables employees to save time and costs while maintaining quality.
The software robots,
called bots in the following, imitate the clicks and tasks on a computer that the employee would have to perform manually.
By delegating these repetitive and recurring tasks to his personal bot, the employee creates space and time to devote to
new and non-trivial tasks that can not be automated.
At first sight this sounds very good and helpful for the employee,b ut the positive advantages, which this technology
offers and which one hopes for, can develop very quickly into the exact opposite, if not sufficient time for a matured
Error Management is invested.\\

The SAP - Digital, Commercial Operations team, which is generally responsible for handling customer support requests,
has implemented several bots.
During the implementation of these bots, no Error Management was implemented for the bots.
The bots are able to work along the happy path in the process, but if an exception or error is thrown in the process,
the bot has no further steps it can take and simply fails.
From the log data collected by the Commercial Operations team over the last few months, it appears that only one out
of seven implemented bots has a success rate of almost 60\%.
The other automations have an even lower success rates.
By implementing a sophisticated error management, the productivity and stability and usability of the bots can be
increased.
In the best case the reliability of the bots should be increased to at least 80\%.
Due to the time limitation of this bachelor thesis, not all seven process automations can be included in this work.
Therefor a concept should be conceived in the context of this Thesis, which can be used in order to apply
Error Management to automations, which were not included in this project.

\section{Objective of the Thesis}\label{sec:objective-of-the-thesis}
In the context of this project, it must be clarified that several parties are involved in this project.\\
\begin{enumerate}
    \item{The university}
    \item{The company}
    \item{The student}
\end{enumerate}
First, let's look at the objectives of each party.\\

\textbf{The company}\\
The company, or more precisely the Commercial Operations department, has the goal of increasing the success rate of
their bots to 80\% with the successful completion of this project.
Since it is not possible to include all automations
in this project due to time constraints, a concept should emerge from this project that can be used to apply Error Management to future automations.\\

\section{Artifacts of the Thesis}\label{sec:artifacts-of-the-thesis}
In the previous chapter: "Objectives of the Thesis" some artifacts were named, which are necessary for the achievement
of the objectives.
In addition, artifacts are listed here, which are used as aids to achieve the goals of all parties involved.\\
\begin{enumerate}
    \item {The log files}
    \item {The Bachelor Thesis}
    \item {Error Management Concept}
    \item {RPA Automations}
\end{enumerate}\\
\textbf{The log files}\\
The log files, which are used as information base in this thesis, are not generated by IRPA - Cloud Studio itself,
because IRPA - Cloud Studio comes with a built-in logging for exceptions and errors.
Information such as time, type and where these errors are thrown are logged by default.
This means that logging of errors does not need to be rebuilt, but the logging must be adapted to the requirements of this project.
For example, analyzing and solving errors, bugs and misbehavior of the bots are a central tasks in this project.
But to replicate and analyze the explicit use case, it is necessary for the developer to include temporary data like
the "OrderID" in the log when throwing such an error message.
This increases the quality of the log data.
The data from one year ago until today is available.\\

\textbf{The Error Management Concept}\\
The ErrorManagement concept is an artifact that is developed during the bachelor thesis by the student.
This concept should show a guideline, after one ErrorManagement with each Bot to apply can.
In general, the concept should divide the process of error management implementation into different phases.
Basically in 3--4 (e.g., process observation, error identification, error investigation and solution implementation).
The user of this concept should work through these phases step by step.
During the "journey" through these phases, technical knowledge shall be imparted, which shall contribute to the error
management understanding.
In addition, the general reasons for certain error messages will be explained.
In addition, it can sometimes happen that no reason for an error can be determined.
But also for this case you can take certain steps to narrow down the problem or to collect more data about this problem.
The concept is responsible for passing on the knowledge that has been accumulated during the work on this thesis and
to be able to benefit from it for their own bots.


\\
\textbf{The RPA Automations}\\
The RPA automations were implemented before the start of this thesis by the Commercial Operations IRPA team and will be
improved in the course of the development of the thesis with the information gained from the log files.
 The bots will be improved in productivity, stability and usability.
 This can be measured and validated by analyzing and comparing the future log data with the data from the previous months.\\
\newpage
\textbf{Project Environment and Conditions (Demographics)}\\
This project is basically considered as an internal project of SAP and Mannheim University of Applied Sciences.
Do SAP and the university with the student represent the steakholders of this project.
The RPA automations are developed by the Commercial Operations IRPA team and used by the support agents as an auxiliary tool.
It is important to note that the IRPA team uses a solution developed by SAP for the implementation of the bots.
SAP IRPA Cloud Studio, therefore, this project is limited by the available functions in IRPA Cloud Studio.
This is important to note because IRPA Cloud Studio does not offer a fully customizable Error Management ao of today(20.10.2021).
For example, there is no possibility to create custom errors in IRPA Cloud Studio.
This means that there are only a few error categories to which you are limited.
For this reason a small workaround has to be made and information that is relevant for the IRPA team to optimize has
to be attached to the few existing error types.
The error is fed with additional data, so to speak.
Another factor that plays a crucial role in this project is the fact that the IRPA bots are all developed and executed
in a closed environment.
In addition, new implementations and improvements can be patched to the production system in just 20 minutes and
theoretically new and up-to-date log data/user data can be obtained the very next day.

In addition it must be still noted here that each agent lets its Bot always work on his or her own computer.
This means that the bots do not run in a standardized runtime environment (e.g., a virtual mashine).
For example there is a Bot, which must start the Internet Explorer on the local computer.
Some agents have their browser settings set by default to automatically redirect all links that are attempted to be
opened through Internet Explorer to Microsoft Edge, and Internet Explorer is closed immediately.\\

\textbf{Analysing scheme}\\
Due to the fact that the project relevant data was implemented without taking Error Managent into consideration, time
should first be invested to review the complete process flow once in the team and to understand and discuss the process
logic.
It is noticeable that at some points in the process there are small error messages that can be easily avoided by making
small corrections in the bot.
This has led to the realization that the log data of the last few months is very abundant, but the information contained
in the log data is very redundant.
Therefore, a small workaround is undertaken to feed the errors with process relevant data and to analyze them at a later
time (e.g., 1 week later) and to implement Error Management successfully.
This way, the quantity of data to be analyzed manually
is reduced, and the quality is significantly improved.
After that you can start to review the log weekly and handle the non-trivial error messages.
If for some reason an error is not traceable or replicable, the error message can be fed with additional data.
In some cases it is even necessary to meet the agent running the bot and try to look at the problem on his computer,
because as mentioned before there are always slightly different system requirements on the computers of the colleagues.
In order to improve the success rate of the bots, this approach must be repeated in several cycles,
like a loop.\\

\textbf{Testing scheme}\\
Of course, it is obvious that if new optimizations have been made in a bot, to test this bot manually on your own
computer before.
If the first test was successful, then the new improvements can be patched to the productive bots.
After some time has passed (e.g.\ a week), the log data for this bot from last week is analyzed for anomalies, as
described in the previous chapter.

In IRPA Cloud Studio there is a debugger to help the developer to test his bot.

In IRPA there is no testing framework that allows the user to write tests for a bot, as it would be possible with JUnit.
\newpage

\section{Table of contents}\label{sec:table-of-contents-v1.1}
\begin{enumerate}
    \item {Introduction}
    \begin{enumerate}
        \item {Motivation}
        \item {Tasks and Goals}
        \item {Scientific question}
        \item {Aufbau der Arbeit}
    \end{enumerate}
    \item {Basic about RPA}
    \begin{enumerate}
        \item {About RPA}
        \item {About IRPA?}
        \item {This chapter can be extended depending on the Topic being discussed in the following chapters }
    \end{enumerate}
    \item {Demographie}
    \begin{enumerate}
        \item {Who is using the Automations?}
        \item {Which automations are used?}
        \item Projekt conditions
        \begin{enumerate}
            \item {whick Software we are Using?}
            \item {Which recources are available?}
            \item {limitations of IRPA}
            \item {Enviromantal curcumstancis}
        \end{enumerate}
    \end{enumerate}
    \begin{enumerate}
        \item {The Problem to solve}
        \begin{enumerate}
            \item {Problem description}
            \item {consequences}
            \item {Workarounds have to be done (könnte auch in ein späteres kapitel)}
        \end{enumerate}
    \end{enumerate}
    \item {Goals and Requirements}
        \begin{enumerate}
            \item Bot Successrate
            \item Error Management Concept
        \end{enumerate}
    \item {Solution Concept}
        \begin{enumerate}
            \item Error Prediction
            \item Discussion
            \item Implementation
            \item Testing
            \item Logdata research
        \end{enumerate}
    \item {Implementation Use Cases}
        \begin{enumerate}
            \item Change Request
            \item Find Crystal Licence Key
            \item Find Invoce of store Order
            \item OppID
            \item Partner Apps ARC Submitted
            \item Partner Apps new User registration
            \item Replicate B1 Order
            \item Termination Flow
        \end{enumerate}
    \item {Validation}
    \begin{enumerate}
        \item Asking the Agents/Team?
        \item Analyse Logdata evolution
        \item Asking for feedback?(maybe do some research)
        \item Generated value ? (messbar?)
        \item Valuation
    \end{enumerate}
    \item {Discussion}
    \item {Alternatives}
    \item {Summary und next Steps}
\end{enumerate}

